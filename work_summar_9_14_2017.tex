\documentclass{article}
\usepackage[utf8]{inputenc}
\usepackage{graphicx}
\usepackage{amsmath}
\usepackage{tabu}
\usepackage{gensymb}
\usepackage{tikz}
\usetikzlibrary{arrows}
\usepackage[export]{adjustbox}
\usepackage[T1]{fontenc}
\usepackage{braket}

%\usepackage[pdftex,active,tightpage]{preview}
%\setlength\PreviewBorder{2mm}
\usepackage{tikz-3dplot}
\usepackage{listings, xcolor}
\usepackage[colorlinks = true,
            linkcolor = blue,
            urlcolor  = blue,
            citecolor = blue,
            anchorcolor = blue]{hyperref}

\title{Fault Profile Generator Tools User Guide}
\author{Sonam Ghosh J82C}
\date{August 2017}
\lstset{
    string=[s]{"}{"},
    stringstyle=\color{blue},
    comment=[l]{:},
    commentstyle=\color{black},
}


%\title{senior_thesis_work_summary_9_14_2017}
%\author{sonamghosh }
%\date{September 2017}

\begin{document}

%\maketitle
Work Progress 9/14/2017 \newline
Note: Images made on MS Paint. Apologies for the bad quality when constructing the diagram. \newline
\begin{figure}[h]
    \centering
    \includegraphics[scale=0.5]{benzene_optical_01.png}
    \caption{Multiport setup in Benzene-like Setup}
    \label{fig:my_label}
\end{figure}

\begin{figure}[!h]
    \centering
    \includegraphics[scale=0.5]{multiport_cell_01.png}
    \caption{Multiport cell close up of one of the segments labeled}
    \label{fig:my_label}
\end{figure}

\newpage
States denoted as:
$$ \ket{n, C, D} $$
Where $n$ is the lattice slice , $n \in \{1,2,3,4,5,6\}$, C is which way around the molecule a photon/electron is rotating either clockwise $CW$ or counter clockwise $CCW$, where I define $CW$ to be $-1$ and $CCW$ to be $1$, $C \in \{-1, 1\}$. D is the motion of the particle, if it is on the side of a single input/output then it is left $L$ and if it is on the double input/output side then it is right $R$, $D \in \{L, R\}$. \newline
Ports $A, B, C$ are noted by their states in a standard basis as follow: 
$$ \ket{A} = \begin{pmatrix} 
1 \\
0 \\
0
\end{pmatrix} ,
 \ket{B} = \begin{pmatrix} 
0 \\
1 \\
0
\end{pmatrix} ,
 \ket{C} = \begin{pmatrix} 
0 \\
0 \\
1
\end{pmatrix} 
$$
States of direction are denoted as follow:
$$ \ket{L} =  \begin{pmatrix} 
1 \\
0 \\
0
\end{pmatrix} , 
\ket{R} = \frac{1}{\sqrt{2}}\begin{pmatrix}
0 \\
1 \\
1
\end{pmatrix}
$$
Available nearest-cell interactions: \newline
\begin{enumerate}
    \item $\ket{n,1,L}\bra{n,1,R}$
    \item $\ket{n,1,R}\bra{n+1,1,R}$
    \item $\ket{n,-1,R}\bra{n,-1,R}$
    \item $\ket{n-1,1,L}\bra{n,1,L}$
    \item $\ket{n-1,-1,L}\bra{n,-1,L}$
    \item $\ket{n,-1,R}\bra{n,-1,L}$
\end{enumerate}
\end{document}

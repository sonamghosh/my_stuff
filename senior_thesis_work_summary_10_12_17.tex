\documentclass{article}
\usepackage[utf8]{inputenc}
\usepackage{graphicx}
\usepackage{amsmath}
\usepackage{tabu}
\usepackage{gensymb}
\usepackage{tikz}
\usetikzlibrary{arrows}
\usepackage[export]{adjustbox}
\usepackage[T1]{fontenc}
\usepackage{braket}

%\usepackage[pdftex,active,tightpage]{preview}
%\setlength\PreviewBorder{2mm}
\usepackage{tikz-3dplot}
\usepackage{listings, xcolor}
\usepackage[colorlinks = true,
            linkcolor = blue,
            urlcolor  = blue,
            citecolor = blue,
            anchorcolor = blue]{hyperref}

\title{Fault Profile Generator Tools User Guide}
\author{Sonam Ghosh J82C}
\date{August 2017}
\lstset{
    string=[s]{"}{"},
    stringstyle=\color{blue},
    comment=[l]{:},
    commentstyle=\color{black},
}


%\title{senior_thesis_work_summary_9_14_2017}
%\author{sonamghosh }
%\date{September 2017}

\begin{document}

Sonam Ghosh Work Summary 10/12/2017 \newline
\newline
State (without considering movement) denoted as:
$$ \ket{n, B}$$
where $n$ is the unit cell number, $n \in \{1,2,3\}$ and $B$ denoting whether the photon is in a double bond region or single bond region, $B \in \{S, D\}$ \newline
$$ \ket{S} = \ket{A} = (1,0,0)^{T} , \ket{D} = 1/\sqrt{2} (\ket{B} + \ket{C}) = 1/\sqrt{2} (0,1,1)^{T}$$ 
\begin{figure}[h]
    \centering
    \includegraphics[scale=0.35]{optical_benzene_diagram_02.png}
    \caption{The three unit cells of optical benzene connected, note L refers to S and R refers to D}
    \label{fig:my_label}
\end{figure}
Available nearest neighbor interactions:
\begin{enumerate}
    \item $D \rightarrow S$, $n$ fixed
    \item $S \rightarrow D$, $n$ fixed
    \item $D \rightarrow S$, $n \rightarrow n-1$
    \item $S \rightarrow D$, $n \rightarrow n+1$
\end{enumerate}
Hamiltonian is given by \newline
\begin{equation}
    \begin{split}
        \hat{H} = \alpha \sum_{n=1}^{3} (\ket{n,D}\bra{n, S}) + \ket{n, S}\bra{n. D}) \\
        - 2 \big( \ket{n, D}\bra{n-1, S} + \ket{n, S}\bra{n+1, D} \big)
    \end{split}
\end{equation}
\newpage
Now consider the motion of the photon as either leftmoving $LM$ or right moving $RM$, our new state will be denoted as 
$$ \ket{n, M, B} $$
where $M$ refers to the motion of the photon, $M \in \{rm, lm\}$
The available interactions are as follow:
\begin{enumerate}
    \item $D \rightarrow S, rm, n$ fixed
    \item $S \rightarrow D, lm, n$ fixed
    \item $S \rightarrow D, rm, n \rightarrow n+1$
    \item $D \rightarrow S, lm, n \rightarrow n-1$
    \item $D \rightarrow D, lm, n$ fixed
    \item $D \rightarrow D, rm, n$ fixed
    \item $S \rightarrow S, lm, n$ fixed
    \item $S \rightarrow S, rm, n$ fixed
\end{enumerate}
The hamiltonian is given as \newline
\begin{equation}
    \begin{split}
        \hat{H} = \alpha \sum_{n=1}^{3} \big( \ket{n, rm, D} \bra{n, rm, S} + \ket{n, lm, S} \bra{n, lm, D} \big) \\
        - 2 \big( \ket{n, rm, S} \bra{n+1, rm, D} + \ket{n, lm, D} \bra{n-1, lm, S} \\
        + \ket{n, lm, D} \bra{n, lm, D} + \ket{n, rm, D} \bra{n, rm, D} \\
        + \ket{n, lm, S} \bra{n, lm, S} + \ket{n, rm, S} \bra{n, rm, S} \big)
    \end{split}
\end{equation}
\end{document}
